
\begin_layout Section
Addressing weaknesses and using strengths
\end_layout

\begin_layout Standard
Initially, practice seems like the first obvious way of approaching my weaknesse
s.
 I am already addressing my reading difficulties by writing short notes
 and holding myself accountable to having a good pace when reading by sharing
 those notes with my supervisor and briefly presenting what I have understood
 to him.
 This also ensures I get some feedback in my reading.
\end_layout

\begin_layout Standard
With respect to the course, since I have an upcoming collaborative paper,
 I intend to apply some of the learned aspects to the general process.
 I will be working with another person at my same level and two or three
 seniors with different specialisations, so that will give good opportunity
 to exercise the feedback and writing process interaction, as well as figure
 out what in the concrete interdisciplinary field I should be focusing on
 (being closer to Mathematics than my prior publication).
 Writing with others will also help me identify more concretely common but
 also diverse ways of formulating ideas (those to be presented not only
 in a formal way).
 Hopefully the course will enable me to gain awareness on how my own feedback
 is taken and processed by others, something I do not have any impression
 of yet.
\end_layout

\begin_layout Standard
On the reading side, some of the points I couldn't address whilst thinking
 about the CARS analysis will hopefully help me narrowing down the style
 and trends in writing in the specific field(s) I will be contributing to.
\end_layout


