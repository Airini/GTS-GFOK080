
\begin_layout Section
Criteria analysis
\end_layout

\begin_layout Standard
This analysis is to be taken from the perspective of having started on a
 PhD position three months ago.
 Hence, my experience in some respects is still under development, which
 is the reason for which I decided to take the introductory course to writing
 in the first place.
 I have, however, already written a paper which got accepted in a peer-reviewed
 venue and am in the process of starting a new (collaborative) one.
\end_layout

\begin_layout Standard
In this section I already start pin-pointing some of the things I perceive
 I do well and others which require development (ie: weak aspects to my
 writing and relation to literature in my field).
\end_layout

\begin_layout Subsection
Critical reading
\end_layout

\begin_layout Standard
I am comfortable in reading different sorts of papers (more theoretical,
 practical, literature surveys, etc.).
 My knowledge of my specific (Computer Science) discipline allows me to
 understand texts quite well, although as a starting PhD there are some
 specific topics (theory-wise) which I am studying to gain full technical
 (mostly pure or mathematical one) understanding.
\end_layout

\begin_layout Standard
Perhaps because of having studies in two of the most relevant disciplines
 (Computer Science and Mathematics) to my field (Functional Programming),
 I am capable of relating concepts in articles published and general trends;
 even identify the research 
\begin_inset Quotes eld
\end_inset

school
\begin_inset Quotes erd
\end_inset

 I am associated to because of my particular interests in the field and
 perspective on problems analysed.
 In general, this gives me a good starting point to read into texts critically
 and compare how the difference in view point affects what I focus on in
 a text or problem with what is highlighted as important in it.
\end_layout

\begin_layout Standard
Despite being comfortable in reading these papers, I am aware that I still
 have to work towards increasing my background knowledge in the more specific
 topics I will work with.
 Additionally, I am currently collaborating and communicating with people
 in other (quite different) fields of engineering; a connection via which
 I am also realising the differences in style and assumptions from field
 to field.
 For instance, these other fields are less explicit in their definition
 of concepts than what I am used to in Computer Science.
\end_layout

\begin_layout Subsection
Audience and genre awareness
\end_layout

\begin_layout Standard
As mentioned, being in a starting phase, I am still constructing the full
 view of potential readers of my work.
\end_layout

\begin_layout Standard
Even though (covered in the following points) there are certain aspects
 to writing I have identified in the communities I will be writing for,
 I still have a long way to go in this respect and still rely on heavy feedback
 from supervisors or other senior people in the field.
 That being said, not only reading is relevant for this question.
 By having attended several conferences and workshops already, spoken and
 discussed with the authors of relevant papers in my field of study already,
 I am already forming an intuition on how to communicate.
 Apart from the individual aspect of writing, I have been making of connections
 which I expect to be responsive in the future if I ask for feedback on
 a specific piece of text falling under their domain of work.
\end_layout

\begin_layout Subsection
Textual coherence and information structure
\end_layout

\begin_layout Standard
For my prior publication, I was quite methodological in deciding how to
 structure the text and how to write.
 I took a few publications from the past two years in that venue, but since
 they were a bit removed in object of study from my particular work, I decided
 to take as a reference an article by my then-supervisor and a couple of
 articles in direct relation to my work (which were amongst of the works
 I cited in the work).
\end_layout

\begin_layout Standard
I took particular attention to the style in writing in my supervisor's article.
 I am aware that I have a tendency to writing overly complex compound sentences
 and knew I had to adjust my general way of expressing ideas to produce
 a more approachable text.
 I consciously try to keep my writing direct and concise (without it being
 incomplete) when writing for my field since it seems to convey a higher
 clarity in ideas and makes for an easier read as well (by own experience
 and short feedback I got, for instance, from my examiner when writing my
 master's thesis).
 Concepts can become quite abstract so clear, definite writing is beneficial
 for readers' understanding and to keep their interest whilst reading.
\end_layout

\begin_layout Standard
In terms of text structure, when writing I aim for each section of the paper
 to stand for itself independently, as far as possible given the dependency
 between ideas introduced.
 For example, I introduce what each section will cover not only at the end
 of the typical 
\begin_inset Quotes eld
\end_inset

Introduction
\begin_inset Quotes erd
\end_inset

 section, but also when starting each section so as for the reader to have
 a clear expectation and be able to keep in mind the main goal of the section
 throughout the process of reading of it.
\end_layout

\begin_layout Standard
The first feedback for my past publication was quite positive with respect
 to the quality of the writing and structure, so I perceive this general
 approach is appropriate for my field of study.
\end_layout

\begin_layout Subsection
Data commentary
\end_layout

\begin_layout Standard
In my particular topics of study, data commentary is not always present
 in the work produced given the theoretical nature of some of the studies
 published.
\end_layout

\begin_layout Standard
I in general haven't identified problems in my own presentation of data
 and commenting of it.
 Since it is most common in my field to use LaTeX editing, naming/labelling
 and referencing data (may it be tables, graphs, algorithms, sample code,
 etc.) is in general quite straight forward.
 In terms of presenting data, if there are several sets to of gathered data
 to comment on, I find it preferable to interleave them with the comments
 referring to them such that each piece is introduced first, presented right
 after and finally analysed or described more concisely (depending on the
 overall structure of the written piece and whether conclusions from results
 are extracted elsewhere) before moving on to the next particular data.
 This way, each piece of evidence or information is encapsulated individually
 and clearly pointed at for the reader.
 But this is subject to the style imposed by the venue of publication (for
 which a styling LaTeX file is normally provided or referenced if a standard
 one is used); with a two-column format imposed this approach might not
 always be preferable.
\end_layout

\begin_layout Subsection
Peer response and writing process
\end_layout

\begin_layout Standard
Frequently I approach the writing process by laying out an initial skeleton
 which I then gradually fill out and adjust as I go, depending on whether
 I realise some concepts should be introduced at different stages to make
 reading the text easier and ensure each section has a well-defined purpose.
 I normally do not give a draft for feedback until I have a well-defined
 initial structure in my mind so as to receive feedback that is more useful
 to the more general idea I have for a specific text.
 But after that, I request feedback regularly with the completion of each
 section so as to incorporate any additional general feedback I can extract
 from particular remarks to the rest of the writing.
\end_layout

\begin_layout Standard
As I am used to doing so for code when programming, I use version control
 systems (eg: git) to track my writing too.
 This makes it easier for someone else, generally a supervisor, to check
 the current state of my writing and also makes it very simple and effortless
 to track the evolution of the text (useful if one wants to backtrack for
 example) without having to explicitly keep past versions of a text.
\end_layout

\begin_layout Standard
I have received different kinds of feedback from supervisors: my past supervisor
 preferred adding remarks on the final processed (pdf) document whilst my
 current supervisor refers to where he finds an issue or has remarks in
 the document via a separate file or e-mail.
 Either way has so far worked for me, but I attribute this to their clarity
 in giving feedback.
 In any case, both approaches follow the same style: pointing at specific
 areas where there might be a conflict.
 General comments on a piece of writing are useful to me if no example of
 what is being pointed at is given.
\end_layout

\begin_layout Standard
When I have been responsible for giving feedback on a text, I have (maybe
 by imitation) followed a similar style to that of my supervisors.
 More particularly, I highlight parts I find disconnected or incomplete
 by asking what they relate to more concretely, or make suggestions in a
 direct way but using modal verbs such as 
\begin_inset Quotes eld
\end_inset

could
\begin_inset Quotes erd
\end_inset

.
 If I find something has not been mentioned, I tend to pose a meaningful
 question about it (by showing what its relevance could be).
 However, in general, I haven't had a chance yet to see whether my feedback
 can have an impact.
\end_layout

\begin_layout Subsection
Language proficiency
\end_layout

\begin_layout Standard
In terms of technical writing and the vocabulary to use, I would initially
 say I do well.
 Perhaps because of my studies in mathematics I have a good intuition with
 respect to which forms of expression are preferable and the resulting texts
 tend to have no ambiguous sections or remarks, which I consider is vital
 when transmitting research ideas and results.
\end_layout

\begin_layout Standard
Occasionally, particularly when indicating possible explanations for experimenta
l phenomena, I do use a vocabulary which does not transmit as much decisiveness
 as a reader would like to perceive.
 For instance, I may abuse the use of the modal verb 
\begin_inset Quotes eld
\end_inset

would
\begin_inset Quotes erd
\end_inset

 to express things conditionally when it is preferred (I infer from remarks
 I have gotten) to state what the supposed condition is and then carry on
 expressing consequences with the more definite-sounding 
\begin_inset Quotes eld
\end_inset

will
\begin_inset Quotes erd
\end_inset

 or 
\begin_inset Quotes eld
\end_inset

does
\begin_inset Quotes erd
\end_inset

.
 These kinds of constructs I also identify as being problematic in others'
 writing, so I pay close attention to it when further re-drafting my writing.
\end_layout

\begin_layout Section
Main strengths
\end_layout

\begin_layout Standard
One of my biggest strengths which has not been brought up yet, is my ease
 to communicate in English language.
 Having studied in lower, middle and high school in English has already
 prepared me to analyse texts in the language and write (more or less) coherentl
y with a thought-out structure.
 This possibly also came with training towards English language examinations
 such as CPE (Proficiency).
\end_layout

\begin_layout Standard
Having a relevant and strong background in two different fields related
 to my research (with a total of seven years of study) is definitely helpful
 too in terms of maturity when it comes to reading comprehension and ability
 to express technical information.
\end_layout

\begin_layout Standard
I am observant when reading, so I extract how each writer decides to present
 their work depending on what their aim is.
 Since I have had some practice with mimicking other people's style and/or
 structuring, I feel confident that I can do similarly for future pieces
 meant for a different venue or particular community slightly different
 to the one I have started my training in.
\end_layout


